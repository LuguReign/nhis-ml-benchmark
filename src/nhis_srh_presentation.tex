% file: presentation/nhis_srh_presentation.tex
\documentclass[aspectratio=169]{beamer}

\usepackage[utf8]{inputenc}
\usepackage[T1]{fontenc}
\usepackage{lmodern}
\usepackage{booktabs}
\usepackage{siunitx}
\usepackage{pgfplots}
\usepackage{pgfplotstable}
\usepackage{tikz}
\usepackage{graphicx}
\usepackage{xcolor}

\usetheme{metropolis} % if not installed, comment out and use default

\pgfplotsset{compat=1.18}
\sisetup{round-mode=places,round-precision=3,detect-all}

% --------- Paths to artifacts (adjust if needed) ----------
\def\ART{artifacts} % directory where CSVs live
\def\IMP{\ART/grouped_importance_l1_adults24.csv}

% --------- Title -----------
\title{NHIS → ML: Self-Rated Health (SRH) Prediction}
\subtitle{Adults23 → Adults24 generalization; grouped permutation importance}
\author{NHIS ML Benchmark (Adults23 train, Adults24 test)}
\date{\today}

% ======= helper: read & prepare importance table =======
% Expect columns: nhis_var, delta_auc, delta_auc_std, pretty_label
\pgfplotstableread[col sep=comma]{\IMP}\importanceRaw
% We assume your CSV is already sorted (desc by delta_auc) by our script.
% If unsure, you can uncomment the next line to sort in TeX by delta_auc descending:
% \pgfplotstablesort[sort key=delta_auc, sort cmp=float >]{\importanceSorted}{\importanceRaw}
\pgfplotstablecopy{\importanceRaw}\importanceSorted

% Count rows
\pgfplotstablegetrowsof{\importanceSorted}
\pgfmathsetmacro{\NROWS}{\pgfplotsretval}

% Limit to Top-15 (or all if fewer)
\pgfmathtruncatemacro{\NTOP}{ min(int(\NROWS),15) }

% Build yticklabels macro for the first \NTOP rows (pretty_label fallback to nhis_var)
\newcommand{\MakeYTicks}{%
  \pgfplotsinvokeforeach{0,...,\numexpr\NTOP-1\relax}{%
    \pgfplotstablegetelem{##1}{pretty_label}\of{\importanceSorted}%
    \edef\_lbl{\pgfplotsretval}%
    \ifx\_lbl\empty
      \pgfplotstablegetelem{##1}{nhis_var}\of{\importanceSorted}%
      \edef\_lbl{\pgfplotsretval}%
    \fi
    \edef\temp{ \_lbl, }%
    \temp
  }%
}

% Build (x=delta_auc, y=index) coordinate list for first \NTOP rows
\newcommand{\PlotCoords}{%
  \pgfplotsinvokeforeach{0,...,\numexpr\NTOP-1\relax}{%
    \pgfplotstablegetelem{##1}{delta_auc}\of{\importanceSorted}%
    \edef\_x{\pgfplotsretval}%
    (\_x,##1)
  }%
}

% ======= Document =======
\begin{document}

\maketitle

% ----------------------------------------------------------
\begin{frame}{Goal and Data}
\textbf{Project goal.} Make NHIS (complex survey) \emph{ML-friendly}: a clear, reproducible prediction task and baseline benchmarks.

\medskip
\textbf{Outcome.} Self-rated health (PHSTAT\_A), binarized: \texttt{Fair/Poor = 1} vs \texttt{Good+ = 0}.

\medskip
\textbf{Setup.}
\begin{itemize}
  \item Train: \textbf{Adults23} (external generalization emphasis)
  \item Test: \textbf{Adults24} (out-of-sample year)
  \item Weights: NHIS weight \texttt{WTFA\_A}, normalized
  \item Predictors: SES, education, employment/benefits, social, disability/functioning, mental health, conditions, chronic, access to care
\end{itemize}
\end{frame}

% ----------------------------------------------------------
\begin{frame}{Modeling Summary (lean baseline)}
\begin{itemize}
  \item End-to-end pipelines in scikit-learn:
  \begin{itemize}
    \item \textbf{Preprocessing}: missing handling; 1/2 $\rightarrow$ 1/0 for binary; ordinal cleanup; OHE for categoricals
    \item \textbf{Models}: Logistic (L1), Random Forest; calibrated versions available
  \end{itemize}
  \item Thresholds tuned on Adults23 using out-of-fold predictions to optimize \textbf{weighted F1}
  \item Evaluation on Adults24: \textbf{AUC, weighted accuracy, balanced accuracy, weighted F1, AP}
\end{itemize}
\end{frame}

% ----------------------------------------------------------
\begin{frame}{Adults24: Test Metrics (your reported results)}
\centering
\small
\begin{tabular}{l c c c c c}
\toprule
Model (thr) & AUC & W. Acc & Bal. Acc & W. F1 & Avg. Prec \\
\midrule
RF (0.55)   & 0.8574 & 0.8512 & 0.7664 & 0.5507 & 0.5581 \\
L1 (0.65)   & 0.8562 & 0.8515 & 0.7658 & 0.5537 & 0.5626 \\
RF-Cal (0.25) & 0.8582 & 0.8415 & 0.7719 & 0.5475 & 0.5579 \\
L1-Cal (0.30) & 0.8561 & 0.8633 & 0.7566 & 0.5555 & 0.5624 \\
\bottomrule
\end{tabular}

\medskip
\textit{Note: L1 baseline AUC on Adults24 reported as 0.8562; thresholds selected via Adults23 OOF.}
\end{frame}

% ----------------------------------------------------------
\begin{frame}{Adults24: Confusion Matrices (at tuned thresholds)}
\small
\begin{columns}[t]
\column{0.5\textwidth}
\textbf{L1 (thr=0.65)}\par
\begin{tabular}{r r r}
\toprule
& \multicolumn{2}{c}{Pred} \\
& 0 & 1 \\
\midrule
True 0 & 23602 & 3672 \\
True 1 & 1787 & 3568 \\
\bottomrule
\end{tabular}

\column{0.5\textwidth}
\textbf{RF (thr=0.55)}\par
\begin{tabular}{r r r}
\toprule
& \multicolumn{2}{c}{Pred} \\
& 0 & 1 \\
\midrule
True 0 & 23535 & 3739 \\
True 1 & 1768 & 3587 \\
\bottomrule
\end{tabular}
\end{columns}

\medskip
\footnotesize
F1 favored recall on the minority class (Fair/Poor), improving balanced accuracy with modest precision trade-off.
\end{frame}

% ----------------------------------------------------------
\begin{frame}{Top Drivers on Adults24 (Grouped Permutation Importance)}
\small
We compute \textbf{grouped permutation importance} at the \emph{original NHIS variable level}: shuffle one raw column at a time, run the full pipeline, and measure the drop in \textbf{weighted AUC}.

\medskip
\textbf{This slide reads directly from} \texttt{\IMP}. If the file is missing, place it at \texttt{\IMP}.
\end{frame}

% ----------------------------------------------------------
\begin{frame}{Top-15 drivers (Adults24, L1) — Bar chart}
\centering
\begin{tikzpicture}
\begin{axis}[
    xbar,
    height=0.62\paperheight,
    width=0.95\paperwidth,
    xmin=0,
    xlabel={$\Delta$ AUC when shuffled},
    ytick=data,
    yticklabels={\MakeYTicks},
    yticklabel style={/pgf/number format/fixed, font=\footnotesize, align=right},
    xticklabel style={/pgf/number format/fixed, font=\footnotesize},
    bar width=5pt,
    enlargelimits=0.02,
    axis line style={draw=none},
    tick style={draw=none},
    xmajorgrids=true,
    grid style={gray!20},
]
\addplot+[draw=none, fill=blue!65] coordinates { \PlotCoords };
\end{axis}
\end{tikzpicture}

\footnotesize
Largest drops typically appear for disability/functioning, cardiometabolic conditions, and poverty ratio.
\end{frame}

% ----------------------------------------------------------
\begin{frame}{Top-15 drivers (Adults24, L1) — Table}
\centering
\scriptsize
\pgfplotstabletypeset[
    columns={pretty_label,nhis_var,delta_auc,delta_auc_std},
    columns/pretty_label/.style={string type,column name=Label},
    columns/nhis_var/.style={string type,column name=NHIS code},
    columns/delta_auc/.style={fixed,precision=4,column name=$\Delta$AUC},
    columns/delta_auc_std/.style={fixed,precision=4,column name=SD},
    every head row/.style={before row=\toprule,after row=\midrule},
    every last row/.style={after row=\bottomrule},
    % print only first \NTOP rows (assumes CSV sorted desc)
    rows={0,...,\numexpr\NTOP-1\relax}
]{\importanceSorted}
\end{frame}

% ----------------------------------------------------------
\begin{frame}{Interpretation (Adults24, L1)}
\small
\textbf{Takeaways.}
\begin{itemize}
  \item \textbf{Disability \& functioning} (DISAB3\_A, DIFF\_A) are the dominant signals for Fair/Poor SRH.
  \item \textbf{Cardiometabolic disease} (HYPEV\_A, DIBEV\_A, CHD/Stroke) contributes next.
  \item \textbf{Socioeconomics} (POVRATTC\_A, food security) and \textbf{education} add consistent but smaller signal.
\end{itemize}

\textbf{Caveats.}
\begin{itemize}
  \item Permutation importance can split importance among correlated variables; treat clusters as the driver.
  \item Results are predictive, not causal; complex survey design is respected via weights, not full design-based variance.
\end{itemize}
\end{frame}

% ----------------------------------------------------------
\begin{frame}{Methodological Notes}
\small
\begin{itemize}
  \item Outcome: ordinal SRH (1--5) binarized to \texttt{Fair/Poor} vs \texttt{Good+} for classification; ordinal modeling is a planned extension.
  \item Adults23 $\rightarrow$ Adults24 ensures temporal generalization.
  \item Weights \texttt{WTFA\_A} are normalized and passed to fitting and scoring.
  \item Thresholds tuned on Adults23 (OOF) to optimize \textbf{weighted F1}; metrics reported on Adults24 at those thresholds.
  \item For interpretability in this deck we use \textbf{grouped permutation importance on raw variables} to avoid OHE label plumbing.
\end{itemize}
\end{frame}

% ----------------------------------------------------------
\begin{frame}{Reproducibility}
\small
\begin{itemize}
  \item Code: Python + PyCharm; parquet inputs \texttt{data/Adults23\_corefeatures.parquet}, \texttt{data/Adults24\_corefeatures.parquet}
  \item Artifacts saved to \texttt{artifacts/}: pipelines (\texttt{*.joblib}), test metrics, predictions, subgroup metrics, and importance CSVs
  \item This PDF reads CSVs at compile time; ensure \texttt{\IMP} exists (Top-15 slide/table)
\end{itemize}
\end{frame}

% ----------------------------------------------------------
\begin{frame}[standout]
Questions?
\end{frame}

\end{document}
